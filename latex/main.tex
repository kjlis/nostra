\pdfoutput=1
\pdfcompresslevel=9
\pdfinfo
{
	/Author (Krzysztof Lis)
	/Title (Projekt systemu wspomagającego ocenę ryzyka kredytowego w oparciu o dane historyczne)
	/Subject (Tematyka)
	/Keywords (Slowa kluczowe)
}
%\documentclass[a4paper,polish,onecolumn,oneside,floatssmall,11pt,titleauthor,wide,openright]{mwrep}
%\usepackage[scale={0.7,0.8},paper=a4paper,twoside]{geometry}
\documentclass[a4paper,onecolumn,oneside,11pt,wide,floatssmall]{mwrep}
%\usepackage{polish}
\usepackage{amsmath}
\usepackage{amsfonts}
\usepackage{amssymb}
\usepackage{amsthm}
\usepackage{bookman}
\usepackage{verbatim}
\usepackage{booktabs}
\usepackage{float}
\usepackage{subcaption}
\usepackage{appendix}
\usepackage{geometry}
\usepackage[utf8x]{inputenc}
\usepackage[T1]{fontenc}
% \usepackage{t1enc}
% \usepackage[pdftex, bookmarks]{hyperref}
\usepackage[pdftex, bookmarks=false]{hyperref}
\def\url#1{{ \tt #1}}

\usepackage{listings}
\usepackage{enumitem}
\usepackage{lscape}
% marginesy
\textwidth\paperwidth
\advance\textwidth -55mm
\oddsidemargin-0.9in
\advance\oddsidemargin 33mm
\evensidemargin-0.9in
\advance\evensidemargin 33mm
\topmargin -1in
\advance\topmargin 25mm
\setlength\textheight{48\baselineskip}
\addtolength\textheight{\topskip}
\marginparwidth15mm

\clubpenalty=10000 % to kara za sierotki
\widowpenalty=10000 % nie pozostawia wdów
\brokenpenalty=10000 % nie dzieli wyrazów pomiędzy stronami
\sloppy

\tolerance4500
\pretolerance250
\hfuzz=1.5pt
\hbadness1450

% ŻYWA PAGINA
\renewcommand{\chaptermark}[1]{\markboth{\scshape\small\bfseries \
		#1}{\small\bfseries \ #1}}
\renewcommand{\sectionmark}[1]{\markboth{\scshape\small\bfseries\thesection.\
		#1}{\small\bfseries\thesection.\ #1}}
\newcommand{\headrulewidth}{0.5pt}
\newcommand{\footrulewidth}{0.pt}
\pagestyle{uheadings}

\usepackage[pdftex]{color,graphicx,xcolor}
\usepackage[polish]{babel}

% \textheight232mm
% \setlength{\textwidth}{\textwidth}
% \setlength{\oddsidemargin}{\evensidemargin}
% \setlength{\evensidemargin}{0.3cm}
\usepackage[sort, compress]{cite}

%\usepackage{multibib}
%\newcites{bk,st,doc,web}{Książki i~artykuły,Standardy i~zalecenia,Dokumentacja produktów,Publikacje i~serwisy internetowe}

\theoremstyle{definition}
\newtheorem{defn}{Definicja}[section]
\newtheorem{conj}{Teza}[section]
\newtheorem{conjmain}{Teza}
\newtheorem{exmp}{Przykład}[section]

\theoremstyle{plain}% default
\newtheorem{thm}{Twierdzenie}[section]
\newtheorem{lem}[thm]{Lemat}
\newtheorem{prop}[thm]{Hipoteza}
\newtheorem*{cor}{Wniosek}

\theoremstyle{remark}
\newtheorem*{rem}{Uwaga}
\newtheorem*{note}{Uwaga}
\newtheorem{case}{Przypadek}

\definecolor{ListingBackground}{rgb}{1,1,1}
\definecolor{keyWords}{rgb}{0.5,0,0.33}
\definecolor{comments}{rgb}{0.4,0.4,0.4}
\linespread{1.5}
\setitemize[0]{itemindent=26pt,leftmargin=0pt,topsep=0pt}
\vspace{-5mm}
%Dodanie możliwości zakresu numerownia w jednym punkcie enumerate
\def\itemrange#1{%
	\addtocounter{enumi}{1}%
	\edef\labelenumi{\theenumi--\noexpand\theenumi.}%
	\addtocounter{enumi}{-1}%
	\addtocounter{enumi}{#1}%
	\item
	\def\labelenumi{\theenumi.}}

\lstset{   
	belowcaptionskip=1\baselineskip,
	breaklines=true,
	frame=L,
	xleftmargin=\parindent,
	language=Scala,
	showstringspaces=false,
	keywordstyle=\bfseries\color{green!40!black},
	commentstyle=\itshape\color{purple!40!black},
	identifierstyle=\color{blue},
	stringstyle=\color{orange},
	numbers=left,
	numberstyle=\small,
	showstringspaces=false  
}
\newcommand*\justify{%
	\fontdimen2\font=0.4em% interword space
	\fontdimen3\font=0.2em% interword stretch
	\fontdimen4\font=0.1em% interword shrink
	\fontdimen7\font=0.1em% extra space
	\hyphenchar\font=`\-% allowing hyphenation
}

\begin{document}
	
	% kody źródłowe wplatane w tekst
	\lstdefinestyle{incode}
	{
		basicstyle={\footnotesize},
		keywordstyle={\bf\footnotesize\color{keyWords}},
		commentstyle={\em\footnotesize\color{comments}},
		numbers=left,
		stepnumber=5,
		firstnumber=1,
		numberfirstline=true,
		numberblanklines=true,
		numberstyle={\sf\tiny},
		numbersep=10pt,
		tabsize=2,
		xleftmargin=17pt,
		framexleftmargin=3pt,
		framexbottommargin=2pt,
		framextopmargin=2pt,
		framexrightmargin=0pt,
		showstringspaces=true,
		backgroundcolor={\color{ListingBackground}},
		extendedchars=true,
		% title=\lstname,
		captionpos=b,
		% abovecaptionskip=1pt,
		% belowcaptionskip=1pt,
		frame=tb,
		framerule=0pt,
	}
	
	% kody źródłowe z podpisem
	\lstdefinestyle{outcode}
	{
		basicstyle={\footnotesize},
		keywordstyle={\bf\footnotesize\color{keyWords}},
		commentstyle={\em\footnotesize\color{comments}},
		numbers=left,
		stepnumber=5,
		firstnumber=1,
		numberfirstline=true,
		numberblanklines=true,
		numberstyle={\sf\tiny},
		numbersep=10pt,
		tabsize=2,
		xleftmargin=17pt,
		framexleftmargin=3pt,
		framexbottommargin=2pt,
		framextopmargin=2pt,
		framexrightmargin=0pt,
		showstringspaces=true,
		backgroundcolor={\color{ListingBackground}},
		extendedchars=true,
		% title=\lstname,
		captionpos=b,
		% abovecaptionskip=1pt,
		% belowcaptionskip=1pt,
		frame=tb,
		framerule=0.1pt,
	}
	
	
	
	\renewcommand*\lstlistingname{Wydruk}
	\renewcommand*\lstlistlistingname{Spis wydruków}
	\renewcommand{\labelitemii}{$\star$}
	\renewcommand*\tablename{Tabela}
	
	\pagenumbering{roman}
	\renewcommand{\baselinestretch}{1.0}
	\raggedbottom
	%\pdfoutput=1
\pdfcompresslevel=9
\pdfinfo
{
	/Author (Krzysztof Lis)
	/Title (Projekt systemu wspomagającego ocenę ryzyka kredytowego dla pożyczek P2P platformy Lending Club)
	/Subject (Praca dyplomowa magisterska)
	/Keywords ()
}
%\documentclass[a4paper,polish,onecolumn,oneside,floatssmall,11pt,titleauthor,wide,openright]{mwrep}
%\usepackage[scale={0.7,0.8},paper=a4paper,twoside]{geometry}
\documentclass[a4paper,onecolumn,oneside,12pt,wide,floatssmall]{mwrep}
% \usepackage{polish}
\usepackage{amsmath}
\usepackage{amsfonts}
\usepackage{amssymb}
\usepackage{amsthm}
\usepackage{bookman}

\usepackage{geometry}
\usepackage[utf8x]{inputenc}
\usepackage[T1]{fontenc}
% \usepackage{t1enc}
% \usepackage[pdftex, bookmarks]{hyperref}
\usepackage[pdftex, bookmarks=false]{hyperref}
\def\url#1{{ \tt #1}}

\usepackage{listings}

% marginesy
\textwidth\paperwidth
\advance\textwidth -44mm
\oddsidemargin-1.2in
\advance\oddsidemargin 33mm
\evensidemargin-0.9in
\advance\evensidemargin 33mm
\topmargin -1.9in
\advance\topmargin 25mm
\setlength\textheight{48\baselineskip}
\addtolength\textheight{\topskip}
\marginparwidth15mm

\clubpenalty=10000 % to kara za sierotki
\widowpenalty=10000 % nie pozostawia wdów
\brokenpenalty=10000 % nie dzieli wyrazów pomiędzy stronami
\sloppy

\tolerance4500
\pretolerance250
\hfuzz=1.5pt
\hbadness1450

% ŻYWA PAGINA
\renewcommand{\chaptermark}[1]{\markboth{\scshape\small\bfseries \
#1}{\small\bfseries \ #1}}
\renewcommand{\sectionmark}[1]{\markboth{\scshape\small\bfseries\thesection.\
#1}{\small\bfseries\thesection.\ #1}}
\newcommand{\headrulewidth}{0.5pt}
\newcommand{\footrulewidth}{0.pt}
\pagestyle{uheadings}

\usepackage[pdftex]{color,graphicx}
\usepackage[polish]{babel}

% \textheight232mm
% \setlength{\textwidth}{\textwidth}
% \setlength{\oddsidemargin}{\evensidemargin}
% \setlength{\evensidemargin}{0.3cm}
\usepackage[sort, compress]{cite}

%\usepackage{multibib}
%\newcites{bk,st,doc,web}{Książki i~artykuły,Standardy i~zalecenia,Dokumentacja produktów,Publikacje i~serwisy internetowe}


\begin{document}


\pagenumbering{roman}
\renewcommand{\baselinestretch}{1.0}
\raggedbottom
\begin{titlepage}
% Strona tytułowa
\vbox to\textheight{\hyphenpenalty=10000
\begin{center}
	\begin{tabular}{p{10cm} p{30cm}}
		\begin{minipage}{10cm}
			\begin{center}
				\ \\
				\ \\
				Politechnika Warszawska \\
				Wydział Elektroniki i Technik Informacyjnych \\
				Instytut Informatyki
			\end{center}
		\end{minipage}
		&
		\begin{minipage}{12cm}
			\begin{flushleft}
				\footnotesize
				Rok akademicki 2015/2016
				\vspace*{2.75\baselineskip}
			\end{flushleft}
		\end{minipage} \\
		\vspace*{1.0\baselineskip}
	\end{tabular}
	\includegraphics[width=4cm]{img/logo_pw2}
	\par\vspace{\smallskipamount}
	\vspace*{2\baselineskip}{\LARGE PRACA DYPLOMOWA MAGISTERSKA\par}
	\vspace{3\baselineskip}{\LARGE\strut Krzysztof Lis\par}
	\vspace*{2\baselineskip}{\huge\bfseries Projekt systemu wspomagającego ocenę ryzyka kredytowego \\w oparciu o dane historyczne\par}
	\vspace*{4\baselineskip}
	\hfill\mbox{}\par\vspace*{\baselineskip}\noindent
	
	\begin{tabular}[b]{@{}p{3cm}@{\ }l@{}}
		{\large\hfill } & {\large }
	\end{tabular}
	\hfill
	\begin{tabular}[b]{@{}l@{}}
		Opiekun pracy: \\[\smallskipamount]
		{\large dr inż. Andrzej Ciemski}
	\end{tabular}\par
	\vspace*{1\baselineskip}
	\begin{tabular}{p{\textwidth}}
		\begin{flushleft}
			\begin{minipage}{7cm}
				Ocena \dotfill
				\par\vspace{1.6\baselineskip}
				\dotfill
				\par\noindent
				\centerline{\footnotesize Podpis Przewodniczącego} \par
				\centerline{\footnotesize Komisji Egzaminu Dyplomowego}\par
			\end{minipage}
		\end{flushleft}
	\end{tabular}
\end{center}}

% Życiorys
\newpage\thispagestyle{empty}
\begin{tabular}{p{4cm} p{9cm}}
	\begin{minipage}{4cm}
		\begin{flushleft}
			\includegraphics[height=5.0cm,width=4.0cm]{img/profile.jpg}
		\end{flushleft}
	\end{minipage}
	&
	\begin{minipage}{12cm}
		\begin{flushleft}
			\par\noindent\vspace{1\baselineskip}
			\begin{tabular}[h]{l r}
				{\normalsize\it Kierunek:} & \ \ \ \ \ \ \ \ \ \ \ \ \ \ \ \ \ \ \ \ \ \ \ \ \ \ \ \ \ \ \ \ \ \ \ \ \ \ \ \ \ \ \ Informatyka
			\end{tabular}
			\par\noindent\vspace{1\baselineskip}
			\begin{tabular}[h]{l r}
				{\normalsize\it Specjalność:} & Inżynieria Systemów Informatycznych
			\end{tabular}
			\par\noindent\vspace{1\baselineskip}
			\begin{tabular}[h]{l r}
				{\normalsize\it Data rozpoczęcia studiów:} & {\normalsize \ \ \ \ \ \ \ \ \ \ \ \ \ \ \ \ \ 24.02.2014 r.}
			\end{tabular}
			\par\noindent\vspace{1\baselineskip}
		\end{flushleft}
	\end{minipage}
\end{tabular}

\vspace*{1\baselineskip}
\begin{center}
{\large\bfseries Życiorys}\par\bigskip
\end{center}

\indent
Nazywam się Krzysztof Lis. Urodziłem się 19 czerwca 1991 roku w Nowym Jorku. W 2010 roku rozpocząłem studia na Wydziale Mechatroniki Politechniki Warszawskiej na kierunku Automatyka i Robotyka. I stopień studiów skończyłem na specjalności Automatyka z wyróżnieniem ``Summa Cum Laude'' przyznawanym przez Rektora Politechniki Warszawskiej. W ramach pracy inżynierskiej współpracowałem z firmą Siemens nad robotyzacją procesu paletyzacji. Podczas pracy nad projektem jako pierwszy w Polsce wykorzystałem środowisko Tecnomatix RobotExpert® wprowadzone przez firmę Siemens. Po uzyskaniu tytułu inżyniera rozpocząłem studia II stopnia na Wydziale Elektroniki i Technik Informacyjnych na kierunku Informatyka. W trakcie studiów pracowałem w warszawskim start-upie projektującym system dla centrum obrabiarkowego oraz rozpocząłem pracę jako programista w dziale technologii banku inwestycyjnego Goldman Sachs, gdzie miałem okazję brać udział w projektach prowadzonych w Polsce, Stanach Zjednoczonych oraz Wielkiej Brytanii.
\par
\vspace{2\baselineskip}
\hfill\parbox{15em}{{\small\dotfill}\\[-.3ex]
\centerline{\footnotesize podpis studenta}}\par
\vspace{3\baselineskip}
\begin{center}
	{\large\bfseries Egzamin dyplomowy} \par\bigskip\bigskip
\end{center}
\par\noindent\vspace{1.5\baselineskip}
Złożył egzamin dyplomowy w dniu \dotfill 20\_\_r.
\par\noindent\vspace{1.5\baselineskip}
z wynikiem \dotfill
\par\noindent\vspace{1.5\baselineskip}
Ogólny wynik studiów \dotfill
\par\noindent\vspace{1.5\baselineskip}
Dodatkowe wnioski i uwagi Komisji \dotfill
\par\noindent\vspace{1.5\baselineskip}
\dotfill



% Streszczenie
\newpage\thispagestyle{empty}
\vspace*{2\baselineskip} 
\begin{center}
	{\large\bfseries Streszczenie}\par\bigskip
\end{center}

{\itshape
Celem pracy magisterskiej jest omówienie teorii zagadnienia oceny ryzyka kredytowego w oparciu o dane historyczne. Przeprowadzono analizę danych pochodzących z platformy Lending Club®, na podstawie której dobrano model uczenia maszynowego obliczający prawdopodobieństwo niespłacenia pożyczki w oparciu o dane historyczne. Wybrany model został wykorzystany jako komponent przykładowej aplikacji web'owej umożliwiającej użytkownikowi przegląd statystyk pożyczek oraz ocenę ryzyka kredytowego dla danej pożyczki. W celu poprawy skalowalności aplikacji, uniezależnienia jej od środowiska w jakim jest uruchamiana oraz przystosowania do pracy w chmurze wykorzystano technologię kontenerów aplikacyjnych. 
}
\vspace*{1\baselineskip}

\noindent{\bf Słowa kluczowe}: {\itshape web, kontenery aplikacyjne, ryzyko kredytowe, systemy wspomagające podejmowanie decyzji, machine learning}
\par
\vspace{4\baselineskip}
\begin{center}
	{\large\bfseries The design of the web application supporting credit risk evaluation based on historic data.}\par\bigskip
\end{center}

{\itshape
This thesis picks up on the topic of the credit risk evaluation based on historic data. The author conducted a thorough analysis of data published by Lending Club in order to select an appropriate machine learning model, that would estimate the probability of default for a given loan. This component became the core of a sample web application, that allows users to view loan statistics and assess the credit risk of a particular loan. The scalability, environment isolation and cloud-compatibility issues were addressed by incorporating the application containers technology into the project.}
\vspace*{1\baselineskip}

\noindent{\bf Keywords}: {\itshape web, application containers, credit risk, decision support systems, machine learning}

\end{titlepage}


\end{document}
	
	\tableofcontents
	% \addcontentsline{toc}{chapter}{{Przedmowa1}{vii}}{vii}
	
	% \chapter*{Spis tablic, rysunków i~wydruków}
	% \listoftables
	% \listoffigures
	% \lstlistoflistings
	
	%\setlength{\baselineskip}{7mm}
	\newpage
	\pagenumbering{arabic}
	\setcounter{page}{1}
	
	\chapter{Wprowadzenie}

\section{Ocena ryzyka kredytowego}
Ocena ryzyka kredytowego jest kluczowym zagadnieniem z punktu widzenia współczesnych gospodarek rynkowych. Istnieje wiele różnych metod i narzędzi wykorzystywanych przez instytucje finansowe do oszacowania zdolności kredytowej partnera biznesowego – współcześnie tym zadaniem zajmują się wyspecjalizowane systemy informatyczne wykorzystujące osiągnięcia z dziedziny eksploracji danych. W ramach tej publikacji zostaną przedstawione podstawowe pojęcia związane z oceną ryzyka kredytowego, metody modelowania tego ryzyka wykorzystywane przez banki oraz zostanie zaprezentowana koncepcja platformy wspomagającej podejmowanie decyzji związanych z problemem oceny ryzyka kredytowego dla pożyczek typu P2P dostarczanych przez platformę Lending Club.

\section{Cel pracy}
Celem pracy jest przedstawienie roli oceny ryzyka kredytowego we współczesnych instytucjach finansowych, omówienie znaczenia tego zagadnienia na podstawie kryzysu na rynkach finansowych w latach 2007-2015 oraz zaproponowanie rozwiązań technologicznych wspomagających proces podejmowania decyzji w tym obszarze. W tym celu dokonano analizy danych pożyczkowych udostępnianych przez firmę Lending Club, na podstawie której dobrano model uczenia maszynowego służący jako model obliczeniowy w przykładowej implementacji systemu wspierającego ocenę ryzyka kredytowego.

\section{Struktura pracy}
Praca składa się z sześciu rozdziałów, z których pierwszy stanowi niniejszy wstęp.
\begin{description}
\item [Rozdział drugi] opisuje teorię zagadnienia oceny ryzyka kredytowego. Omówiona została rola działów zajmujących się tym problem w instytucjach finansowych, a także przeprowadzono analizę kryzysu na rynkach finansowych w latach 2007-2015, jako przykładu konsekwencji błędnego modelowania tego ryzyka. 
\item [Rozdział trzeci] został poświęcony analizie danych pożyczkowych udostępnianych przez platformę Lending Club oferującą pożyczki w systemie \textit{peer-to-peer}. Na podstawie przeprowadzonej analizy  wybrano statystyki prezentowane w ramach implementowanego systemu oraz dobrano model uczenia maszynowego, wyliczający prawdopodonieństwo deflacji pożyczki.
\item [W rozdziale czwartym] przedstawiono i porównano 3 rodzaje modeli służących do obliczania regresji logistycznej. W oparciu o analizę danych opisaną w poprzednim dziale dobrano parametry, na podstawie których przeprowadzono proces uczenia modeli, a następnie wybrano najlepszy z nich z uwzględniem środowiska aplikacji webowej, w jakiej będzie osadzony. 
\item [Rozdział piąty] omawia zagadnienie konteneryzacji aplikacji rozproszonych na przykładzie rozwiązania Docker, które  zostało wykorzystane przy implementacji systemu. Technologia ta zostałą wykorzystana ze względu na możliwość odizolowania aplikacji od środowiska serwera na jakim jest uruchamiania, co jest szczególnie istotne w przypadku aplikacji webowych uruchamianych w chmurze.
\item [W rozdziale szóstym] przedstawiono przykładową implementację systemu wspomagającego ocenę ryzyka kredytowego, opartego o przeprowadzoną wcześniej analizę. Omówione zostały nowoczesne technologie wykorzystane przy tworzeniu tej aplikacji.
\item [Rozdział siódmy] jest ostatnim rozdziałem pracy zawierającym jej podsumowanie oraz wnioski końcowe. 
\end{description}
 
	\chapter{Ryzyko kredytowe}

\section{Rola oceny ryzyka kredytowego we spółczesnych gospodarkach rynkowych}

\subsection{Wprowadzenie}

Wraz z postępem globalizacji, międzynarodowe rynki finansowe utworzyły strukturę przypominającą sieć naczyń połączonych. Zdarzenia występujące nawet w obrębie gospodarek pojedynczych państw mogą mieć znamienny wpływ na ogólnoświatową kondycję ekonomiczną. Biorąc pod uwagę, że stan współczesnych rynków ekonomicznych jest silnie skorelowany z kondycją instytucji finansowych takich jak banki, których główna działalność opiera się na kredytowaniu podmiotów zarówno prywatnych i instytucjonalnych, można jednoznacznie stwierdzić, że prawidłowa i precyzyjna ocena ryzyka kredytowego jest rzeczą absolutnie niezbędną z punktu widzenia zachowania równowagi na rynkach finansowych.

Przykładem katastrofalnych skutków, jakie niesie za sobą błąd w modelowaniu tego typu ryzyka jest wielki kryzys finansowy przypadający na lata 2007 – 2015, który zostanie dokładniej opisany w dalszej części artykułu. Ze względu na istotność i wagę tego problemu istnieje szereg dyrektyw międzynarodowych nakładających stosowne regulacje na instytucje finansowe, aby stosowane przez nie praktyki w sferze kredytowania nie mogły doprowadzić do poważnych zachwiań w gospodarkach światowych na miarę wspomnianego kryzysu.

Najbardziej istotnymi dokumentami tego typu są normy Basel II i Basel III wprowadzone przez Basel Comittee on Banking Supervision (BCBS) – organ powołany przez banki centralne sprawujący szeroko pojęty nadzór nad praktykami całego sektora bankowego.

\subsection{Definicja ryzyka kredytowego}

Ryzykiem kredytowym nazywamy ryzyko zaistnienia takich okoliczności, z powodu których wartość portfela kredytów i papierów wartościowych instytucji ulegnie wahaniom (spadkowi) z powodu nagłych i nieprzewidzianych zmian zdolności kredytowej kredytobiorców lub partnerów inwestycyjnych.

Podstawowym podziałem ryzyka kredytowego jest:
\begin{itemize}
\item Ryzyko niewypłacalności, czyli wystąpienia opóźneń bądź całkowitego zaprzestania spłaty kredytu
\item Ryzyko spadku zdolności kredytowej (Credit Ratings)
\end{itemize}

Z punktu widzenia projektowania platformy wspomagającej ocenę ryzyka kredytowego istotna jest ta druga podgrupa.

\subsection{Metody oceny ryzyka kredytowego}

Kluczową miarą z punktu widzenia zarządzania ryzykiemr kredytowym jest RWA - Risk-weighted Assets. RWA składa się z:

\begin{itemize}
\item Sumy wag ryzyka pomnożonych przez kapitał pozycji uwzględnainej w bilansie
\item Sumy wag ryzyka pomnożonych przez
ekwiwalent kredytu dla pozycji nie uwzględnianej w bilansie (np. instrumenty pochodne)
\end{itemize}

\begin{equation} \label{eq:RWA}
RWA = \sum_{i=1}^{N}\alpha_{i}E_{i} + \sum_{j=1}^{M}w_{j}C_{j}
\end{equation}
Gdzie:
\begin{itemize}
	\item $\alpha_{i}$ - waga ryzyka dla i-tej pozycji uwzględnianej w bilansie
	\item $E_{i}$ - kapitał i-tej pozycji uwzględnianej w bilansie
	\item $w_{j}$ - waga ryzyka dla j-tej pozycji nie uwzględnianej w bilansie
	\item $C_{j}$ - ekwiwalent kredytu dla j-tej pozycji nie uwzględnianej w bilansie
\end{itemize}

Formuła (\ref{eq:RWA}) jest najbardziej ogólną formą modelu służącego bankom do wyznaczenia wartości RWA. Następnie w oparciu o tę wartość wyznaczany jest wymagany kapitał zabezpieczający udzielone kredyty – stanowi on 8\% RWA. Istnieją dwa podstawowe podejścia do zagadnienia modelowania ryzyka:

\begin{itemize}
\item Podejście standaryzowane (ang. \textit{Standardized Approach})
\item Podejście oparte na wewnętrznym ratingu (ang. \textit{Internal-Rating Based Approach})
	\begin{itemize}
	\item Foundation IRB
	\item Advanced IRB
	\end{itemize}
\end{itemize}

\subsubsection{Standardized approach}

Wartości wag do wzoru (\ref{eq:RWA}) dostarcza krajowy regulator. Są one zdefiniowane dla poszczególnych klas kredytobiorców, np. korporacja, bank, instytucja rządowa. Jest to najprostsza forma wyznaczania wartości RWA stosowana jedynie przez mniejsze instytucje, które nie mogą sobie pozwolić na stosowanie bardziej zaawansowanych metod modelowania ryzyka.

\subsubsection{Internal-RatingBasedApproach}
W przypadku wariantu “Foundation” bank musi wyznaczyć jedynie prawdopodobieństwo niewypłacalności (ang. \textit{Probability of Default}), natomiast w przypadku metody zaawansowanej (Advanced IRB) oprócz PD bank musi wyznaczyć stratę przy (całkowitej) niewypłacalności (ang. \textit{Exposure at Default}) oraz faktyczną stratę (ang. Loss Given Default), która określa procentowy stosunek poniesionych strat do straty przy niewypłacalności (EAD). Otrzymane wartości pozwalają na wyznaczenie wartości RWA w oparciu wewnętrzy model banku, pod warunkiem, że jego specyfikacja została dostarczona krajowemu regulatorowi oraz pomyślnie przeszła weryfikację poprawności i zgodności z obowiązującymi normami.

\section{Kryzys na światowych rynkach finansowych po roku 2007}

\subsection{Geneza}

Genezy gólnoświatowego kryzysu gospodarczego rynków finansowych należy szukać na rynku pożyczek wysokiego ryzyka udzielanych w USA osobom o niewystarczających możliwościach finansowych (ang. subprime mortgage). Pożyczki te były wykorzystywane jako zabezpieczenie obligacji strukturyzowanych emitowanych przez instytucje finansowe – w szczególności duże banki. Z powodu hossy na rynku nieruchomości i wprowadzenia mniej restrykcyjnych przepisów prawnych, instytucje ratingowe wystawiały rzeczonym obligacjom zawyżone oceny. Ponadto na rynku amerykańskim występowały silne naciski polityczne – ze strony administracji Billa Clintona, a następnie George’a W. Busha – mające na celu rozszerzenie grona potencjalnych kredytobiorców o osoby znajdujące się w gorszej sytuacji ekonomicznej.

\subsection{Przebieg}

Skutkiem nadmiernego rozluźniania polityki pieniężnej było wystąpienie symptomów przegrzania amerykańskiej gospodarki. W obawie przed gwałtownym wzrostem inflacji FED zdecydował się na podniesienie stóp procentowych – z poziomu 1\% w 2004 roku do 5,25\% w 2006 roku. Przełożyło się to bezpośrednio na zwiększenie odsetek, jakimi byli obciążeni kredytobiorcy. Doprowadziło to do zachwiania na rynku nieruchomości – wiele osób traktowało taki zakup jako inwestycję finansowaną za pomocą kredytu. W momencie kiedy spadł popyt zaczęły spadać ceny nieruchomości, co stało się przyczynkiem problemów kredytobiorców. Banki, w obliczu zaprzestania spłat kredytów, zaczęły zajmować hipoteki i masowo sprzedawać przejęte nieruchomości. Poskutkowało to gwałtowną akceleracją spadku cen i spiralnym nawarstwianiem się problemów gospodarki amerykańskiej.

Instytucje finansowe poniosły gigantyczne straty, a wiele z nich stanęło na skraju bankructwa. Rząd amerykański zdecydował się dokapitalizować największe banki w obawie przed kryzysem porównywalnym do kryzysu stulecia z lat 30. Jeden z największych banków inwestycyjnych na świecie, Lehman Brothers, otrzymał odmowę przyznania pomocy od banku centralnego USA i ogłosił upadłość. Zasiało to panikę na rynku finansowym i zapoczątkowało serię nerwowych i pochopnych decyzji, które doprowadziły między innymi to dodruku taniego pieniądza, wzrostu inflacji i deregulacji rynku finansowego.

\subsection{Skutki kryzysu}

Oprócz gwałtownego spadku cen (20 – 30\%) na rynku nieruchomości w Stanach Zjednoczonych, spowolnienie gospodarcze dotknęło wielu krajów na świecie. Brak zaufania wobec instytucji finansowych spowodował ucieczkę inwestorów w kierunku innych branż oraz falę spekulacji.

\begin{figure}[h] \centering %H if want to get it exaclty here
	\includegraphics{img/gdp_rate_2007.png}
	\caption{Wzrost PKB w poszczególnych państwach w 2007 roku.\cite{CIA}}
	\label{abcCode}
\end{figure}

Wzajemna nieufność instytucji finansowych poskutkował wstrzymaniem akcji kredytowej i zamrożeniem inwestycji. Światowe gospodarki pogrążyły się w recesji, czyli charakteryzował je ujemny przyrost PKB (kurczenie się gospodarki). Najmniej zostały dotknięte kraje arabskie, które w dużej mierze bazują na eksporcie surowców.

\subsubsection{Wpływ na rynek surowców naturalnych}

\begin{figure}[h] \centering %H if want to get it exaclty here
	\includegraphics[scale=0.4]{img/fuel_index.png}
	\caption{Indeks cen paliw kopalnych latach 2001 - 2016.\cite{indexmundi}} %indexmundi.com
	\label{fuelPriceIndex}
\end{figure}

\begin{figure}[h] \centering %H if want to get it exaclty here
	\includegraphics[scale=0.4]{img/metal_index.png}
	\caption{Indeks cen metali latach 2001 - 2016.\cite{indexmundi}} %indexmundi.com
	\label{metalPriceIndex}
\end{figure}

Wykresy przedstawione na rysunkachh 2.2 i 2.3 ukazują gwałtowny wzrost zainteresowania inwestycjami opartymi na surowcach naturalnych. Doprowadziło to do nienaturalnego, lecz chwilowego wzrostu cen - cena baryłki ropy wzrosła niemalże trzykrotnie na przestrzeni 18 miesięcy. Niestety ze względu na zamrożenie wielu inwestycji spowodowane brakiem kredytów po skoku cen nastąpił równie raptowny ich spadek. Przysporzyło to wielu problemów przedsiębiorstwom wydobywczym, m.in. doprowadziło do bankructwa wiele australijskich kopalni niklu.

\subsubsection{Wpływ na ceny żywności}

\begin{figure}[h] \centering %H if want to get it exaclty here
	\includegraphics[scale=0.4]{img/food_index.png}
	\caption{Indeks cen żywności latach 2001 - 2016.\cite{indexmundi}} %indexmundi.com
	\label{foodPriceIndex}
\end{figure}

Jednym z tragicznych następstw kryzysu na rynkach finansowych był radykalny wzrost cen żywności. Najbardziej odczuły to kraje rozwijające się i tzw. Kraje Trzeciego Świata. Doprowadziło to do politycznej i ekonomicznej niestabilności w tych regionach.

\subsubsection{Bezrobocie}

Wzajemna nieufność instytucji finansowych poskutkował wstrzymaniem akcji kredytowej i zamrożenia inwestycji. Światowe gospodarki pogrążyły się w recesji, czyli charakteryzował je ujemny przyrost PKB (kurczenie się gospodarki). Najmniej zostały dotknięte kraje arabskie, które w dużej mierze bazują na eksporcie surowców.
za masowymi zwolnieniami w sektorze finansowym (w samej Wielkiej Brytanii pracę w tym sektorze straciło kilkadziesiąt tysięcy osób) ucierpiały również inne branże – przede wszystkim branża motoryzacyjna. Było to spowodowane załamaniem sprzedaży nowych samochodów finansowanej głównie za pomocą kredytów – a banki przestały ich udzielać.
Zadłużenie społeczeństwa i rosnące bezrobocie spowodowały spadek konsumpcji, co w sposób spiralny prowadziło do dalszego spadku produkcji i redukcji zatrudnienia.

\subsection{Walka z kryzysem}

Walka z kryzysem w dużej mierze sprowadzała się do dotowania instytucji sektora finansowego z pieniędzy podatników oraz stosowania praktyk protekcjonistycznych w stosunku do rodzimych firm – szczególnie dotyczyło to USA i Francji. W celu pobudzenia akcji kredytowej wprowadzono rekordowe obniżki stóp procentowych, co miało skłonić inwestorów do bardziej śmiałych i zdecydowanych działań.
Postanowiono wprowadzić nową normę: Basel III, która miała uchronić sektor finansowy przed popełnieniem tych samych błędów. Należy jednak pamiętać, że poprzednia odsłona tych regulacji – mylnie obwiniana za zaistniały kryzys – została ogłoszona w roku 2004, a w roku 2008 dopiero była wdrażana, czyli nie miała szansy realnie wpłynąć na kondycję ekonomiczną sektora bankowego i uchronić świata przed nadchodzącym gospodarczym tąpnięciem.
Rządy państw grupy G8 przekazały w sumie ponad 3 bln. \$ na cel ratowania własnych gospodarek. Postanowiono również zwiększyć rezerwy Międzynarodowego Funduszu Walutowego, a także planowano likwidację rajów podatkowych – pomysł ten nie doczekał się jednak realizacji.
	\input{5_Docker}

	\bibliographystyle{unsrt}
	\bibliography{sample} 
	\begin{appendices}
		\chapter{Załączniki}
		\begin{description}
			\item \textbf{RxConcurrency} \\
			Aplikacja służąca do obliczenia liczby Pi z dokładnością do podanej liczby miejsc po przecinku. \\
			Pokazująca możliwości paradygmatu reaktywnego w kwestiach współbieżności. \\
			Wykorzystane narzędzia: ReactiveX
			\item \textbf{RxPath} \\
			Aplikacja służąca do rysowania ścieżki na ekranie. \\
			Pokazująca możliwości paradygmatu reaktywnego w obsłudze zdarzeń. \\
			Wykorzystane narzędzia: ReactiveX, SodiumFRP
			\item \textbf{RxProcessing}\\ 
			Aplikacja służąca do zliczania linii z literą 'a' i 'b' w podanym pliku.\\
			Pokazująca możliwości paradygmatu reaktywnego w przetwarzaniu danych. \\
			Wykorzystane narzędzia: ReactiveX, Apache Spark
			\item \textbf{RxSolarsystem} \\
			Aplikacja symulująca Układ Słoneczny. \\
			Pokazująca nowe możliwości w tworzeniu architektury oprogramowania.  \\
			Wykorzystane narzędzia: ReactiveX
		\end{description}
	\end{appendices}
\end{document}

% ex: set tabstop=4 shiftwidth=4 softtabstop=4 noexpandtab fileformat=unix filetype=tex spelllang=pl,en spell:






