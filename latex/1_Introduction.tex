\chapter{Wprowadzenie}

\section{Ocena ryzyka kredytowego}
Ocena ryzyka kredytowego jest kluczowym zagadnieniem z punktu widzenia współczesnych gospodarek rynkowych. Istnieje wiele różnych metod i narzędzi wykorzystywanych przez instytucje finansowe do oszacowania zdolności kredytowej partnera biznesowego – współcześnie tym zadaniem zajmują się wyspecjalizowane systemy informatyczne wykorzystujące osiągnięcia z dziedziny eksploracji danych. W ramach tej publikacji zostaną przedstawione podstawowe pojęcia związane z oceną ryzyka kredytowego, metody modelowania tego ryzyka wykorzystywane przez banki oraz zostanie zaprezentowana koncepcja platformy wspomagającej podejmowanie decyzji związanych z problemem oceny ryzyka kredytowego dla pożyczek typu P2P dostarczanych przez platformę Lending Club.

\section{Cel pracy}
Celem pracy jest przedstawienie roli oceny ryzyka kredytowego we współczesnych instytucjach finansowych, omówienie znaczenia tego zagadnienia na podstawie kryzysu na rynkach finansowych w latach 2007-2015 oraz zaproponowanie rozwiązań technologicznych wspomagających proces podejmowania decyzji w tym obszarze. W tym celu dokonano analizy danych pożyczkowych udostępnianych przez firmę Lending Club, na podstawie której dobrano model uczenia maszynowego służący jako model obliczeniowy w przykładowej implementacji systemu wspierającego ocenę ryzyka kredytowego.

\section{Struktura pracy}
Praca składa się z sześciu rozdziałów, z których pierwszy stanowi niniejszy wstęp.
\begin{description}
\item [Rozdział drugi] opisuje teorię zagadnienia oceny ryzyka kredytowego. Omówiona została rola działów zajmujących się tym problem w instytucjach finansowych, a także przeprowadzono analizę kryzysu na rynkach finansowych w latach 2007-2015, jako przykładu konsekwencji błędnego modelowania tego ryzyka. 
\item [Rozdział trzeci] został poświęcony analizie danych pożyczkowych udostępnianych przez platformę Lending Club oferującą pożyczki w systemie \textit{peer-to-peer}. Na podstawie przeprowadzonej analizy  wybrano statystyki prezentowane w ramach implementowanego systemu oraz dobrano model uczenia maszynowego, wyliczający prawdopodonieństwo deflacji pożyczki.
\item [W rozdziale czwartym] przedstawiono i porównano 3 rodzaje modeli służących do obliczania regresji logistycznej. W oparciu o analizę danych opisaną w poprzednim dziale dobrano parametry, na podstawie których przeprowadzono proces uczenia modeli, a następnie wybrano najlepszy z nich z uwzględniem środowiska aplikacji webowej, w jakiej będzie osadzony. 
\item [Rozdział piąty] omawia zagadnienie konteneryzacji aplikacji rozproszonych na przykładzie rozwiązania Docker, które  zostało wykorzystane przy implementacji systemu. Technologia ta zostałą wykorzystana ze względu na możliwość odizolowania aplikacji od środowiska serwera na jakim jest uruchamiania, co jest szczególnie istotne w przypadku aplikacji webowych uruchamianych w chmurze.
\item [W rozdziale szóstym] przedstawiono przykładową implementację systemu wspomagającego ocenę ryzyka kredytowego, opartego o przeprowadzoną wcześniej analizę. Omówione zostały nowoczesne technologie wykorzystane przy tworzeniu tej aplikacji.
\item [Rozdział siódmy] jest ostatnim rozdziałem pracy zawierającym jej podsumowanie oraz wnioski końcowe. 
\end{description}
 